\documentclass[12pt,-letter paper]{article}
\usepackage{siunitx}
\usepackage{setspace}
\usepackage{gensymb}
\usepackage{xcolor}
\usepackage{caption}
%\usepackage{subcaption}
\doublespacing
\singlespacing
\usepackage[none]{hyphenat}
\usepackage{amssymb}
\usepackage{relsize}
\usepackage[cmex10]{amsmath}
\usepackage{mathtools}
\usepackage{amsmath}
\usepackage{commath}
\usepackage{amsthm}
\interdisplaylinepenalty=2500
%\savesymbol{iint}
\usepackage{txfonts}
%\restoresymbol{TXF}{iint}
\usepackage{wasysym}
\usepackage{amsthm}
\usepackage{mathrsfs}
\usepackage{txfonts}
\let\vec\mathbf{}
\usepackage{stfloats}
\usepackage{float}
\usepackage{cite}
\usepackage{cases}
\usepackage{subfig}
%\usepackage{xtab}
\usepackage{longtable}
\usepackage{multirow}
%\usepackage{algorithm}
\usepackage{amssymb}
%\usepackage{algpseudocode}
\usepackage{enumitem}
\usepackage{mathtools}
%\usepackage{eenrc}
%\usepackage[framemethod=tikz]{mdframed}
\usepackage{listings}
%\usepackage{listings}
\usepackage[latin1]{inputenc}
%%\usepackage{color}{   
%%\usepackage{lscape}
\usepackage{textcomp}
\usepackage{titling}
\usepackage{hyperref}
%\usepackage{fulbigskip}   
\usepackage{tikz}
\usepackage{graphicx}
\lstset{
  frame=single,
  breaklines=true
}
\let\vec\mathbf{}
\usepackage{enumitem}
\usepackage{graphicx}
\usepackage{siunitx}
\let\vec\mathbf{}
\usepackage{enumitem}
\usepackage{graphicx}
\usepackage{enumitem}
\usepackage{tfrupee}
\usepackage{amsmath}
\usepackage{amssymb}
\usepackage{mwe} % for blindtext and example-image-a in example
\usepackage{wrapfig}
\providecommand{\mydet}[1]{\ensuremath{\begin{vmatrix}#1\end{\vmatrix}}}
\providecommand{\myvec}[1]{\ensuremath{\begin{bmatrix}#1\end{b\matrix}}}
\providecommand{\cbrak}[1]{\ensuremath{\left\{#1\right\}}}
\providecommand{\brak}[1]{\ensuremath{\left(#1\right)}}
\providecommand{\sbrak}[1]{\ensuremath{{}\left[#1\right]}}
\title{\textbf{Mathematics}}
\date{}
\begin{document}
\maketitle{}
\begin{center}
\section*{Algebra}
\end{center}
\text{Questions:}
\begin{enumerate}
\item Find the value of p, for which one root of the quadratic equation $px^{2}+ 14x + 8 = 0$ is 6 times the other.
\item If $ad \neq bc$, then prove that the equation
	\begin{align}
		\brak{a^2 + b^2}x^2 + 2\brak{ac + bd}x+ \brak{c^2 + d^2} = 0
	\end{align}
    has no real roots.
\end{enumerate}
\begin{center}
\section*{Geometry}
\end{center}
\begin{enumerate}
	\item If the angle between two tangents drawn from an external point $P$ to a circle of radius a and centre $O$, is $60{\degree}$, then find the length of $OP$.
\item A circle touches all the four sides of a quadrilateral $ABCD$. Prove that $AB + CD = BC + DA$.
\item A line intersects the $y$-axis and $x$-axis at the points $P$ and $Q$ respectively. If $\brak{2, -5}$ is the midpoint of $PQ$, then find the coordinates of $P$ and $Q$.
\item In what ratio does the point $\brak{\frac{24}{11}, y}$ divide the line segment joining the points $P\brak{2, -2}$ and $Q\brak{3, 7}$? Also, find the value of $y$.
\end{enumerate}
\begin{center}
\section*{Circles}
\end{center}
\begin{enumerate}                                        
\item Three semicircles each of diameter $3 \,\mathrm{cm}$, a circle of diameter $4\cdot5\,\mathrm{cm}$ and a semicircle of radius $4\cdot5\,\mathrm{cm}$ are drawn in the given figure. Find the area of the shaded region.
\begin{figure}[H]                                     
\centering                                          
\includegraphics[width=\columnwidth]{16.jpg}  
\caption{}
\end{figure}  
\item In the given figure, two concentric circles with centre $O$ have radii $21 \,\mathrm{cm}$ and $42 \,\mathrm{cm}$. If $\angle AOB = 60{\degree}$, find the area of the shaded region.\\
	\sbrak{use 	\pi = \frac{22}{7}}
\begin{figure}[H]      
\centering                                               
\includegraphics[width=\columnwidth]{17.jpg}   
\caption{}
\end{figure}
\end{enumerate}
\begin{center}
\section*{Menstrutations}
\end{center}
\begin{enumerate}
    \item Water in a canal, $5\cdot4\,\mathrm{m}$ wide and $1\cdot8\,\mathrm{m}$ deep, is flowing with a speed of $25\,\mathrm{km/hour}$. How much area can it irrigate in $40\,\mathrm{minutes}$ , if $10\,\mathrm{cm}$ of standing water is required for irrigation ?
\item The slant height of a frustum of a cone is $4 \,\mathrm{cm}$ and the perimeters of its circular ends are $18 \,\mathrm{cm}$ and $6 \,\mathrm{cm}$. Find the curved surface area of the frustum.
\end{enumerate}
\begin{center}
\section*{Probability}
\end{center}
\begin{enumerate}
\item A bag contains 15 white and some black balls. If the probability of 
drawing a black ball from the bag is thrice that of drawing a white ball, 
find the number of black balls in the bag.
\item The probability of selecting a rotten apple randomly from a heap of 900 apples is $0\cdot18\,\mathrm{}$.What is the number of rotten apples in the heap ?
\end{enumerate}
\begin{center}
\section*{Trigonometry}
\end{center}
\begin{enumerate}
\item If a tower $30\,\mathrm{m}$ high, casts a shadow $10\sqrt{3}$
m long on the ground, then what is the angle of elevation of the sun?
\item On a straight line passing through the foot of a tower, two points $C$ and $D$ are at distances of $4\,\mathrm{m}$ and $16\,\mathrm{m}$ from the foot respectively. If the angles of elevation from $C$ and $D$ of the top of the tower are complementary, then find the height of the tower.
\end{enumerate}
\begin{center}
	\section*{Progressions}
\end{center}
\begin{enumerate}
	\item What is the common difference of an A.P. in which $ a_{21} - a_{7} = 84?$
\item Which term of the progression $20,19\frac{1}{4}, 18\frac{1}{2}, 17\frac{3}{4},\ldots$ is the first negative term?
\item The first term of an A.P. is 5, the last term is 45 and the sum of all its 
terms is 400. Find the number of terms and the common difference of the A.P.
\end{enumerate}
\end{document}
